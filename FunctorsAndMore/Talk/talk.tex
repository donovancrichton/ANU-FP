\documentclass[hyperref={colorlinks = true,linkcolor = blue, citecolor = blue, urlcolor = blue}]{beamer}

\setbeamertemplate{navigation symbols}{}
\setbeamertemplate{footline}[frame number]
 
\usepackage[utf8]{inputenc}

\usepackage[newfloat]{minted}

\usepackage{pgf}
\usepackage{tikz}
\usepackage{upquote}
\usepackage{natbib}
\usetikzlibrary{arrows,automata}

\bibliographystyle{abbrvnat}


\newenvironment{code}{\captionsetup{type=listing}}{}
\SetupFloatingEnvironment{listing}{name=Listing}

\title{Functors, Applicatives, Monoids, and Monads)}
\author{Donovan Crichton}
\date{August 2022}

\begin{document}
 
\frame{\titlepage}

\begin{frame}[fragile]
  \frametitle{Preliminaries}
  \begin{itemize}
  \item Slides and Examples available at:
    \url{https://github.com/donovancrichton/ANU-FP}
  \item This talk: /FunctorsAndMore
  \end{itemize}
\end{frame}

\begin{frame}[fragile]
  \frametitle{What is a type class?}
  recall:
  \begin{minted}{haskell}
class Num a where
  (+) :: a -> a -> a
  (-) :: a -> a -> a
  (*) :: a -> a -> a
  negate :: a -> a
  abs :: a -> a
  signum :: a -> a
  fromInteger :: Integer -> a
  \end{minted}
\end{frame}

\begin{frame}[fragile]
  \frametitle{Type classes for programmers}
  \begin{itemize}
  \item A way to introduce ad-hoc polymorphism to our functions.
  \item Specify a 'class' of behaviours for particular types.
  \item Allow operator and function overloading (e.g + on Double and Int)
  \item We can use any type that \emph{subscribes} to the class.
  \end{itemize}
  \begin{minted}{haskell}
f :: Num a => a -> a
f x = x + x
  \end{minted}

\end{frame}

\begin{frame}[fragile]
  \frametitle{Type classes for mathematicians}
  \begin{itemize}
  \item A way to define our own algebra! (almost)
  \item Recall that an algebra is a triple $(A, F, L)$
  \item $A$ is our carrier set.
  \item $F$ is a set of operations on $A$.
  \item $L$ is our set of axioms or laws for $F$.
  \item This is why ':doc' in Haskell gives us something interesting.
  \end{itemize}
\end{frame}

\begin{frame}[fragile]
  \frametitle{Functors in Haskell}
  \begin{minted}{bash}
Prelude> :doc Functor
 The 'Functor' class is used for types that can be mapped 
 over. Instances of 'Functor' should satisfy the 
 following laws:

> fmap id  ==  id
> fmap (f . g)  ==  fmap f . fmap g

The instances of 'Functor' for lists, 
'Data.Maybe.Maybe' and 'System.IO.IO'
satisfy these laws.
Prelude> :info Functor
class Functor (f :: * -> *) where
  fmap :: (a -> b) -> f a -> f b
  (<$) :: a -> f b -> f a
  {-# MINIMAL fmap #-}
  \end{minted}
\end{frame}

\begin{frame}[fragile]
  \frametitle{What is a functor?}
  \begin{itemize}
  \item A morphism between categories (of course!)
  \item Inspired by functors from category theory.
  \item A structure preserving map from objects in FA to objects in FB
  \item $map : (a \overset{g}{\rightarrow} b) \rightarrow F(a) 
         \overset{F(g)}{\rightarrow} F(b)$ where $F$ is a functor.
  \end{itemize}
\end{frame}

\begin{frame}[fragile]
  \frametitle{Functors in Idris}
  \begin{minted}{bash}
interface Prelude.Functor : (Type -> Type) -> Type
Functors allow a uniform action over a parameterised type.
@ f a parameterised type
Parameters: f
Constructor: MkFunctor
Methods:
  map : (a -> b) -> f a -> f b
    Apply a function across everything of type 'a' in a 
      parameterised type
    @ f the parameterised type
    @ func the function to apply
  \end{minted}
\end{frame}

\begin{frame}[fragile]
  \frametitle{Functors as an algebra}
  let $map \text{ (also called $\langle \$ \rangle$)}: (a \overset{g}{\rightarrow} b) \rightarrow F(a) 
         \overset{F(g)}{\rightarrow} F(b)$ where $F$ is a functor. \\ \\
  let $F= (A, \{map\}, \{map(id) = id, map(g \circ h) = map(g) \circ map(h)\})$ \\
  \\
  So we have our carrier set $A$, one operation $map$, and two axioms/laws.

\end{frame}

\begin{frame}[fragile]
  \frametitle{Legal Functors in Idris}
  Here we show a demo on a legal Functor definition in Idris with an example
  candidate implementation on lists.
\end{frame}

\begin{frame}[fragile]
  \frametitle{Applicative Functors}
  An Applicative Functor $F$ is another common algebra over a carrier set $A$.
  With one binary operation and one unary operation: 
  \begin{align*}
    &pure : A \rightarrow F(A) \\
    &apply \text{ (also called $\langle \ast \rangle$)} : F(A \overset{g}{\rightarrow} B) \rightarrow F(A) 
               \overset{F(g)}{\rightarrow} F(B)
  \end{align*}
   And our set of axioms:
   \begin{align*}
    \text{\color{orange}{Identity:}}\\
    apply(pure(id), v) &= v \\
    \text{\color{orange}{Composition:}}\\
    apply(pure(\circ,apply(u, apply(v, w))) &= apply(u,(apply(v, w))) \\
    \text{\color{orange}{Homomorphism:}}\\
    apply(pure(f), pure(x)) &= pure(f(x)) \\
    \text{\color{orange}{Interchange:}}\\
    apply(u, pure(y)) &= apply(pure(\lambda f.f(y)),u) 
   \end{align*}
\end{frame}

\begin{frame}[fragile]
  \frametitle{Applicative Functors 2}
  Our Algebra for Applicative Functors is formally: \\
  Let $F = (A,\{pure,\langle \ast \rangle>\},\{id, com, hom, int\})$ \\
  \\
  Why is the Applicative algebra also a Functor? \\
  We can define $map$ as follows: \\
  $g <\$> x = pure(g) \langle \ast \rangle x$ \\
  \\
  Let's see the demo!
\end{frame}

% \begin{frame}[fragile]
% \frametitle{References}
% \bibliography{references}{}
% \end{frame}

\end{document}

