\documentclass[hyperref={colorlinks = true,linkcolor = blue, citecolor = blue, urlcolor = blue}]{beamer}

\setbeamertemplate{navigation symbols}{}
\setbeamertemplate{footline}[frame number]
 
\usepackage[utf8]{inputenc}

\usepackage[newfloat]{minted}

\usepackage{pgf}
\usepackage{tikz}
\usepackage{upquote}
\usepackage{natbib}
\usetikzlibrary{arrows,automata}

\bibliographystyle{abbrvnat}


\newenvironment{code}{\captionsetup{type=listing}}{}
\SetupFloatingEnvironment{listing}{name=Listing}

\title{An Introduction To Idris \\ \tiny (for the future.)}
\author{Donovan Crichton}
\date{August 2022}

\begin{document}
 
\frame{\titlepage}

\begin{frame}[fragile]
  \frametitle{Preliminaries}
  \begin{itemize}
  \item Slides and Examples available at:
    \url{https://github.com/donovancrichton/ANU-FP}
  \item This talk: /IntroToIdrisAndGADTS
  \end{itemize}
\end{frame}

\begin{frame}[fragile]
\frametitle{Generalised Algebraic Data Types}
\begin{itemize}
  \item Guarded Recursive DataTypes - HOAS \citep{xi2003guarded}.
  \item First Class Phantom Types - Type Equality from Pattern Matching \citep{cheney2003first}.
  \item Generalised Algebraic Data Types - Type Inference \citep{jones2004wobbly}.
\end{itemize}
\end{frame}

\begin{frame}[fragile]
  \frametitle{Syntax}
  What do GADTs actually look like?
  \begin{minted}{haskell}
    {-# LANGUAGE GADTs #-}
    -- this is the data constructor
    data Expr a where
    -- these are the type constructors:
      EInt :: Int -> Expr Int
      EBool :: Bool -> Expr Bool
      EAdd :: Expr Int -> Expr Int -> Expr Int
      EAnd :: Expr Bool -> Expr Bool -> Expr Bool
  \end{minted}
\end{frame}

\begin{frame}[fragile]
  \frametitle{Limitations of Algebraic Data Types}
  Imagine a list with a concrete constructor and a polymorphic one.
  \begin{minted}{haskell}
    -- this is just fine.
    data List a =
      Nil
      | Cons a (List a)
      | CInt Int (List Int)

    -- this is not! 
    tl :: List a -> List a
    tl Nil = Nil
    tl (Cons x xs) = xs
    tl (CInt k ks) = ks
  \end{minted}
\end{frame}

\begin{frame}[fragile]
\frametitle{Limitations of Algebraic Data Types 2}
\begin{itemize}
  \item If we \emph{parameterise} our data types we can't fix that parameter later.
  \item In PLT, TT and Logic we have object and meta languages.
  \item We would like to use the binding in our meta language for our object language.
\end{itemize}
\end{frame}

\begin{frame}[fragile]
\frametitle{Polymorphic Evaluation}
  \begin{minted}{haskell}
    {-# Langauge GADTs #-}
    data Expr a where
      EInt :: Int -> Expr Int
      EBool :: Bool -> Expr Bool
      EAdd :: Expr Int -> Expr Int -> Expr Int
      EAnd :: Expr Bool -> Expr Bool -> Expr Bool

    eval :: Expr a -> a
    eval (EInt x) = x
    eval (EBool b) = b
    eval (EAdd x y) = eval x + eval y
    eval (EAnd p q) = eval p && eval q
  \end{minted}
\end{frame}

\begin{frame}[fragile]
\frametitle{Correctness by Construction}
\begin{itemize}
  \item GADTs now type-check our object language expressions in our meta language.
  \item It is \emph{not possible} for me to accidentally write addition on Boolean expressions.
  \item Idea - create your data structures in such away that they enforce correctness?.
\end{itemize}
\end{frame}

\begin{frame}[fragile]
  \frametitle{Moving Away from Haskell}
  \begin{itemize}
    \item Extensions overload: PolyKinds, DataKinds, KindSignatures, KindAsType, etc.
    \item Reification and Reflection: Haskell has two languages!
    \item Just to get more expressivity and functionality for GADTS?
  \end{itemize}
\end{frame}

\begin{frame}[fragile]
  \frametitle{Stephanie Weirich's Dependent Haskell...}
  Haskell looks different with just \emph{one} extension...
  \begin{minted}{haskell}
  {-# LANGUAGE DataKinds, TypeFamilies, PolyKinds, 
      TypeInType, GADTs, RankNTypes, ScopedTypeVariables,
      TypeApplications, TemplateHaskell, 
      UndecidableInstances, InstanceSigs, 
      MultiParamTypeClasses, TypeOperators, 
      KindSignatures, TypeFamiliyDependencies, 
      AllowAmbiguousTypes, FlexibleContexts, 
      FlexibleInstances #-}
  \end{minted}
\end{frame}

\begin{frame}[fragile]
  \frametitle{To Idris!}
  \begin{itemize}
    \item Full Support for Dependent Types.
    \item Pure Functional Language.
    \item Syntactically Similar to Haskell.
    \item Term and Type Languages are Identical.
    \item It has a book \citep{brady2017type}.
    \item My Favourite.
  \end{itemize}
\end{frame}

\begin{frame}[fragile]
  \frametitle{Installation (on Linux)}
  The README on the repo has good installation instructions.
  \begin{enumerate}
    \item \colorbox{lightgray}{\$ sudo apt-get install chezscheme9,5}
    \item \colorbox{lightgray}{\$ git clone https://github.com/idris-lang/idris2}
    \item \colorbox{lightgray}{\$ cd idris2}
    \item \colorbox{lightgray}{\$ make bootstrap SCHEME=chezscheme9.5}
    \item \colorbox{lightgray}{\$ make install}
    \item \colorbox{lightgray}{\$ make clean}
    \item \colorbox{lightgray}{\$ make all}
    \item \colorbox{lightgray}{\$ make install}
    \item \colorbox{lightgray}{add ~/.idris2/bin to your system path}
  \end{enumerate}
\end{frame}

\begin{frame}[fragile]
  \frametitle{DDTS (GADTS) in Idris}
  Strictness, loss of type inference.
  \begin{minted}{idris}
    data Expr : (a : Type) -> Type where
      EVal : a -> Expr a
      EAdd : Num a => Expr a -> Expr a -> Expr a
      EAnd : Expr Bool -> Lazy (Expr Bool) -> Expr Bool

    eval : Expr a -> a
    eval (Eval x) = x
    eval (EAdd x y) = (eval x) + (eval y)
    eval (EAnd p q) = (eval p) && (eval q)
  \end{minted}
\end{frame}

\begin{frame}[fragile]
  \frametitle{Slim instead of Thick. Fat instead of Thin. Full of holes.}
  \begin{itemize}
    \item \mintinline{idris}{:} not \mintinline{haskell}{::}
    \item \mintinline{idris}{(x :: xs)} not \mintinline{haskell}{(x : xs)}
    \item \mintinline{idris}{=>} not \mintinline{haskell}{->}
    \item \mintinline{idris}{?what}
  \end{itemize}
  \begin{minted}{idris}
    f : Nat -> Nat
    f x =
      case x of
       Z     => ?baseCaseHole
       (S k) => ?stepCaseHole
  \end{minted}
\end{frame}

\begin{frame}[fragile]
  \frametitle{Gentle Dependent Types 1}
  \begin{itemize}
    \item We can \emph{index} our data types as well as \emph{parameterise} them.
    \item The index may have a specific element in the return type, the parameter may not.
    \item in \mintinline{idris}{List}, \mintinline{idris}{a} is a parameter. In 
             \mintinline{idris}{Expr}, \mintinline{idris}{a} is an index.
  \end{itemize}
\end{frame}

\begin{frame}[fragile]
  \frametitle{Gentle Dependent Types 2}
  We can be \emph{precise} about head and tail.
  \begin{minted}{idris}
    data Vect : Nat -> Type -> Type
      Nil  : Vect Z a
      (::) : a -> Vect k a -> Vect (S k) a

    hd : Vect (S k) a -> a      
    hd (x :: _) = x      
      
    tail : Vect (S k) a -> Vect k a      
    tail (_ :: xs) = xs    
  \end{minted}
\end{frame}

\begin{frame}[fragile]
  \frametitle{Scary Dependent Types 1}
  \begin{minted}{idris}
    data (=) : a -> b -> Type where
      Refl : a = a
  \end{minted}
  We can state equality between any two types. However we can only construct
  an element of the equality type if \mintinline{idris}{a} and \mintinline{idris}{b}
  are the same thing. \\ \\
  This happens by beta reduction.
\end{frame}

\begin{frame}[fragile]
  \frametitle{Scary Dependent Types 2}
  Interactive Demo on defining (++)
\end{frame}

\begin{frame}[fragile]
  \begin{itemize}
    \item What is rewrite? Goal changing?
    \item Also implicits?
    \item More proofs like this, more practice? Port from Coq!
    \item Read \href{https://softwarefoundations.cis.upenn.edu/lf-current/toc.html}{logical foundations} and
           skip the tactics!
  \end{itemize}
\end{frame}

\begin{frame}[fragile]
  \frametitle{Okay...but that's not a proof!}
    \begin{itemize}
      \item A proof based on intuitionistic higher order logic.
      \item Thanks to the curry-howard isomorphism.
      \item Commonly known as 'Propositions as Types' \citep{wadler2015propositions}.
      \item As Phillip Wadler will now \href{https://www.youtube.com/watch?v=IOiZatlZtGU}{show us}!
    \end{itemize}
\end{frame}

\begin{frame}[fragile]
  \frametitle{Next Week}
  \begin{itemize}
    \item Functor, Applicatives, Monoids and Monads!
    \item IO (Finally!)
    \item suggestions? (Plan: Parser Combinators, FRP, Lenses, and more)
  \end{itemize}
\end{frame}

\begin{frame}[fragile]
\frametitle{References}
\bibliography{references}{}
\end{frame}


\end{document}

